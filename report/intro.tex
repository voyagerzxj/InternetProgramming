\section{Introduction}
\subsection{Introduction}

The hospital information management system is an indispensable information management system for today's medical institutions. It can provide sufficient space and a really quick way to search prescription for each medical institution to manage patient and physician information, so that it is greatly convenient for medical institutions. With the proper management of managers, hospital information management system is very useful for the managers of medical institutions.

With the rapid development of science and technology, computer science has also become more and more mature, almost all offices are benefit from computers. Computers have occupied a very high position in human life and work. In the past, prescriptions were mostly handwritten, complicated and time-consuming, and after the prescriptions were generated, it was not easy to save them, which required a lot of manpower and material resources, also it was inconvenient to find them. Different from traditional manual input, electronic prescriptions are simple and fast to use, have high reliability, large storage capacity, good confidentiality, long life, and low cost. It is conducive for the management of patient's basic data and tracking. At the same time, it can also be used to find information about the patient's past medical treatment and the corresponding departments and doctors, which can simplify the process of seeking medical treatment again. These advantages can greatly improve the efficiency of patient and physician management.
\subsection{The significance of development}

Nowadays, the popularity of computers has been really high. At the same time, its performance has also been qualitatively improved, thus it is used in various fields, which has been indispensable in our learning and working life. Nowadays, with the improvements of hospital’s scale and scope of business, simple labor control can no longer meet the needs of current medical institutions. At the same time, manual operations will inevitably lead to errors. As far as the current medical resources are concerned, the ordinary manpower cannot satisfy the growing number of patients. Thus it can be seen that the previous manual mode has not been adapted to the development of the times. It not only wastes a lot of manpower and material resources, but also delays the treatment of patients. Therefore, this traditional management method will be replaced by information management is an inevitable result. 

As a team from Computer Science major, we hope that through our own efforts to develop an electronic hospital information management system that can solve the above problems, to strengthen the hospital management, improve the quality of medical care, and provide a reliable solution for the hospital which can safely and efficiently store patient information over the years. When there is a demand, we can quickly find out all kinds of information for patients and physicians. This can effectively help medical institutions manage patients and physicians’ data.